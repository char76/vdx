% --------------------------------------------------------------------------
% 
% Benutzerhandbuch f�r den View Designer/X
%
%	in Englisch
% 
% --------------------------------------------------------------------------


\documentstyle[epsf,palatino,twoside,makeidx,refman]{report} 

\tolerance 400%
\emergencystretch 1em%
%\hfuzz .2
%\vfuzz\hfuzz

%\sloppy


% -- A5 --
%\setlength\textheight {165mm}
%\setlength\fullwidth  {110mm}
%\setleftmarginwidth{2.5cm}

\setlength\textheight {8.3in}

\setleftmarginwidth{4cm}

\makeindex

\def\glossarname{Glossary}

\def\lazyglossar{ \newpage
	      \addcontentsline{toc}{chapter}{\glossarname}
	\longthickhrule\bigskip
     {\secshape\Large\bf\sf \glossarname \par}
     \bigskip\longthickhrule\bigskip
    \markboth{\uppercase{\glossarname}}{\uppercase{\glossarname}}%
    \thispagestyle{plain}
}

\def\lazyindex{
	\newpage
	      \addcontentsline{toc}{chapter}{\indexname}
	\printindex
}

% ------------------------------------
% Zapfding-Font
% ------------------------------------
\font\symbolfont=pzdr at 14pt

\def\arrow{{$\rightarrow$}}

\newenvironment{mylist}[1]{ \begin{list}{}{
	\listparindent0cm \itemindent0cm \labelsep0.1cm
	\settowidth{\labelwidth}{\bf #1} 
	\setlength{\leftmargin}{\labelwidth}
	\addtolength{\leftmargin}{\labelsep}	
	\renewcommand{\makelabel}[1]{##1 \hfill}
}}{\end{list}}

\newenvironment{lazy}{\bigskip\hrulefill\begin{alltt}\small}{\end{alltt}\hrulefill\bigskip}

\def\syntax#1{{\em \marginlabel{Syntax:}}{\tt #1}}
\def\desc#1{{\em \marginlabel{Description:}}{#1}}
\def\function#1#2{\syntax{#1}

\desc{#2}}

\def\myitem#1{\marginlabel{\em #1}}

\def\showkey#1{{\tt <#1>}}

\newtheorem{myexample}{Example:}[chapter]
\newenvironment{exam}{\smallskip\begin{myexample}\rm \ }{\hfill$\Box$\end{example}}


\def\epsfsize#1#2{\ifnum#1>\hsize\hsize\else#1\fi}

\newcommand{\psfigure}[2] {
	\begin{figure} 
	\begin{center}
	\ \epsfbox{#1}
	\end{center}
	\caption{\label{#1} #2}
	\end{figure}
}

\newcommand{\psmargin}[1] {
	\marginpar{
	\begin{center}
	\ \epsffile{#1}
	\end{center}
	}
}

\newcommand{\marginchar}[1] {
	\parskip1mm
	\marginpar{
		\begin{center}
			{\myfont #1}
		\end{center}
	}
}

\def\telefon{\marginchar{\char37}}
\def\brief{\marginchar{\char41}}
\def\finger{\marginchar{\char43}}
\def\haken{\marginchar{\char51}}


% ------------------------------------
% headings
% ------------------------------------

\headheight 1cm

% ------------------------------------
% masse
% ------------------------------------

\parindent0.0cm		% Keine Einrueckung beim Absatzbeginn
%\footheight0.0cm
%\footskip0.0cm
%\textheight20.5cm
%\textwidth14cm
%\headwidth14cm
%\oddsidemargin1cm
%\evensidemargin1cm

\pagestyle{headings}

% ------------------------------------
% Beginn des Dokumentes
% ------------------------------------

\begin{document}

% ------------------------------------
% nocite
% ------------------------------------

\pagenumbering{roman}

	\hfill \ \epsffile{logo_new.ps}
	\ \vskip14cm

	{\Large \sf
	\hfill View Designer/X \par
	\hfill User Guide \par
	}
	\hfill {\small \sf Release 1.1} \par

	\smallskip
	\longhrule
	\smallskip
	\hfill {\small \sf \today} \par

\thispagestyle{empty} 
\newpage
%\maketitle

{\small
Copyright {\copyright} 1996, 1997, Dirk L�ssig

\bigskip

	UNIX and Motif are trademarks of The Open Group.
}

\newpage

\tableofcontents				  	% Inhaltsverzeichnis
\newpage

\chapter{Introduction}

\pagenumbering{arabic}

\section{The Goal of View Designer}

	Using X Window System and Motif toolkit it's possible to create user
	friendly applications. Without support of powerful tools this is a hard
	work for application developers.

	The View Designer is such a tool which helps software engineers to
	develop graphical user interfaces for applications running on UNIX
	systems. 

\section{Features Provided by Release 1.1}


	The latest release of View Designer provides following features:

	\begin{itemize}

	\item Object oriented handling including Drag \& Drop
	\item Interactive WYSIWYG view
	\item Widget Resource Editor
	\item Widget Tree Browser
	\item Templates of widget tree
	\item Reusable components 
	\item Generating C and C++ code
	\item Adaption of code generation

	\end{itemize}


\section{Usage of this User Guide}

	For better understanding of View Designer, knowledge of C and
	C++ programming and Motif toolkit is needed. If you don't have such
	knowledge you should read a book that introduces Motif programming
	prior to reading this document.

	Following conventions are used throughout this user guide:

	\begin{itemize}
	\item \myitem{Example}In the left column of a page you see important
	words which refers to contents of the text. These could be used for
	rapid finding of searched explanations.

	\item If we refer to keyboard usage the following convention is used:
	\showkey{F1}. That means you have to press the first function key on
	your keyboard. 

	\item Important words are highlighted with an {\em italic}
	font. Source code sequences are shown in {\tt typewriter} font.
	\end{itemize}

	In the following chapters we introduce parts of user 
	interface of View Designer. After this we explain the 
	design process with an example. In the last chapters some special
	features of the View Designer are shown. But at first the next 
	section gives an overview of the functionality. 

\section{Basic Concepts}

	\myitem{\bf Projects}View Designer supports the development of Motif
	applications. It works with a project that consists of interfaces and
	several settings.

	\psfigure{overview_en.eps}{Overview}

	\myitem{\bf Interfaces}An interface contains one widget tree, i.e.
	widgets that are ordered hierarchically. Widget are graphical user
	interface components used to create Motif user interfaces. A widget
	belongs to a widget class and has several settings like name and
	resources. Resources are responsible for appearance and behavior of
	widgets. 

	View Designer supports design of interfaces in an interactive
	way. The tools of View Designer create, edit and delete widgets.
	The WYSIWYG view shows the final appearance of a user interface. The
	hierarchy of widgets can be changed using Widget Tree Browser and the
	properties of widgets are edited by Widget Resource Editor. Other
	helpful tools simplify the creation of interfaces, e.g. Templates.

	Interfaces are saved in files with extension {\tt .i}. The source code
	generation creates a few files for every interface containing code to
	create final user interface.

	\myitem{\bf Project setting}A project has many properties, e.g. the
	name of the application, source code generation model, name of 
	a interface, that appears first when the application starts up,
	and compiler and linker options. These
	properties and names of containing interfaces are saved in a project
	file ({\tt .prj}).


\chapter{User Interface of View Designer}

	This chapter introduces the user interface of View Designer.
	After reading it
	you will be familiar with tools and functionality. In the following
	chapter knowledge is improved with an example.

\section{Main Window}

	% Menus
	\index{Main Window}
	Figure \ref{mainwindow.eps} shows the main window of View Designer. It
	is the starting point for creation of user interfaces in Motif and
	generation of source code.

	% Bild

	\psfigure{mainwindow.eps}{Main Window}


	At the top of main window a menu bar is placed. Projects can be
	created, loaded, changed and saved by using the contained menus.
	More windows of View Designer are reachable by this menus. The global
	user configuration can be changed by this menus. Also available is an
	online help system.


	\myitem{\bf File}
	The file menu contains items which create, read or save projects. A
	project contains a set of interfaces and settings. The final
	application is made by generating code and executing {\tt make}.
	If you want to build a new application, you have to create a project.
	Now follows the description of the file menu:


	\myitem{New}
		A new project is created. A little dialog comes up and you have
		to enter the name of the new project. The project file name
		and the name of final executable is derived from this name.
		If you are currently working on a project and there are
		unsaved changes, you will be asked, whether 
		you want to throw away the changes.

	\myitem{Open}\index{project file}Choose this menu item to load a
		formerly saved project. A file selection box is displayed 
		and you
		have to specify a project file name. This file name has always
		the extension {\tt .prj}. Not only the project file is 
		belonging
		to a project. Also interface files with extension {\tt .i} are
		loaded. The project file only refers to these file. Interface
		files are always in the same directory as the project file.


	\myitem{Save}This Menu item is used to save the current project
		to file system. If you save the project the first time a file
		selection box is shown. You have to specify the project file
		name. All interfaces are saved in the same directory.
	
	\myitem{Save as}A file selection box is shown and you have to specify
		the name of the project file. After clicking on Ok button the
		project is saved.

	\myitem{New Templates}\index{Templates}This menu items creates a new
		template folder, which is appearing as a new window. Using
		this window you can create new templates and save the template
		folder in a template file ({\tt .pal}). Templates are widget
		trees with predefined resources.

	\myitem{Open Templates}The template file is opened using menu item
		{\em Open Templates}. In a file selection box you have to
		specify the file name with extension {\tt .pal}. If the
		template file is successfully opened, template folder is
		displayed.


	\myitem{Exit}This menu item closes the application. If there are
		changes on the current project, you will be ask
		whether to abandon or save them. Answering
		'Yes' causes the View Designer to save the project in the
		auto-save directory (see later). So you are able to restore it
		the next time. If you don't want to exit the View Designer
		without normal saving, you have to answer 'No'.


	\myitem{\bf Project Menu}
	In Project menu you are able to apply actions to current project.
	These actions are e.g. changing settings and generating source code.

	\myitem{Settings}\index{Project Settings}Using this menu item settings
		of current project can be changed. Settings are e.g. the
		project name, the path where generated source code is placed,
		name of startup interface and how code is generated.
	
	\myitem{Import Interface}\index{Interface!importing}After choosing this
		menu item a file selection box is shown. If an interface file
		with extension {\tt .i} specified and you press OK, this file
		is inserted into current project. The imported interface is a
		part of project and could be changed.

	\myitem{Generate}\index{Source Code!generating}To generate the source
		code for a project, choose menu item {\em Generate} and a
		little dialog will be shown. This dialog enables you to
		specify which source file you want to overwrite. From Imakefile
		the {\tt xmkmf} utility creates a makefile, the main program
		contains the function {\tt main()} that initializes
		application and opens the startup interface. The callbacks of
		an user interface are contained in source files which are
		different to those containing user interface creation. These
		files are generated by View Designer the first time. If you
		want to overwrite these source files select {\em Stubs}. After
		pressing {\em OK} files are generated but VDX don't write
		the files. A dialog containing a list is used to give you a
		chance for decision of what files you really want to write.

	
	\myitem{Execute}\index{xmkmf}\index{make}This menu item is used for
		execution of {\tt xmkmf}, {\tt make depend}, {\tt make} and
		final application. In a dialog you choose one of these
		programs and press {\em Apply}. The output of this process is
		written into output text pane of main window. Only one program
		can be executed. When it's running a push button labeled with
		stop icon is appearing. View Designer is sending Signal {\tt
		SIGKILL} to running application, if you click on stop icon.

	\myitem{\bf Widgets Menu}\index{Widget!creating}To create widgets
		for an user interface you should open the widget class palette
		or use sub-menus in {\em Widgets} menu.

	\myitem{Palette}
		Using this menu item you get a palette of all available widget
		classes. The widget classes are appearing as icons. To create
		a widget click on one of these icons.

		Sub-menus in {\em Widgets} menu could also be used for menu
		creation. The available widget classes are divided into
		Shells, Managers, Primitives and Dialogs. If you've chosen a
		widget class the mouse pointer is grabbed and you have to click
		and drag the mouse. 

		
	\myitem{{\bf Options Menu}}\index{Options}View Designer is
		configurable. These configuration options are saved into file
		{\tt .vdxrc} in your home directory. The configuration is done
		by items in {\tt Options} menu.

	\myitem{Generic}\index{Generic Options}This opens a dialog for generic
		options configuration. Following properties of View Designer
		can be changed:
	
		\begin{itemize}
		\item The {\em Quick Start Dialog} presents short cut
		operations for users at program startup. When VDX is started
		the dialog is shown and offers three actions: Open, New and
		Recover Project. If this dialog is wanted use {\em Generic}
		options. 

		\item A project is saved automaticly if {\em Auto Saving} is
		on. The interval between these auto savings is adjusted by
		{\em Auto Save Time} slider. The directory where project is
		saved can be changed with {\em Auto Save Path} text entry.

		\item {\em Main Template Folder} is a file that is used as the
		main template palette. The file has extension {\tt .pal}. If
		user wants another palette than default, the file name can be
		changed to refer to his own template file.

		\item A {\em Quick Help} line is placed in bottom of every
		window. If these helpful lines are unwanted change {\em Show
		Lines} toggle button's state to of.

		\item When a project is loaded from a file the WYSIWYG View is
		automaticly shown if this {\em WYSIWYG after load} toggle
		button is set.
		\end{itemize}


	\myitem{Grid}\index{Grid}If widgets are created, resized and moved,
		a grid is used to support convenient placing. Default value of
		grid are ten pixels. To change this, the dialog, that is opened
		by {\em Grid} menu item, could be used.

	\myitem{Update Policy}\index{WYSIWYG}To control the update behavior
		of WYSIWYG view this sub-menu contains two items. If {\em
		Minimum} is chosen, the refresh of child widget doesn't refresh
		the parent. Otherwise, e.g. refresh of a child of XmRowColumn
		causes XmRowColumn to refresh. This behavior is called the
		{\em Optimum}. To refresh widgets explicitly, choose menu item
		{\em Refresh} in widget's context menu.

	\myitem{Default Shell}\index{Default Shell}If a widget is created on
		the root window, an interface is created implicitly. The
		top widget of an interface is always a shell. Creating a
		non-shell widget on root window causes the insertion of
		default shell of a class that is specified by sub-menu {\em
		Default Shell}

	\myitem{View Defaults}\index{View Defaults}The default view settings
		are changeable by this dialog. Here the user is able to specify
		default view parameter like {\em Small Icons}. For Widget Tree
		Browser and Menu Editor the icon views are adjustable. For all
		views the {\em Automatic Load} options can be changed.


	\myitem{Default Public}\index{Default Public}If you create an
		application some resources, e.g. labelString, should always be
		set in external resource file (public). That's important to
		design your 
		application for different languages. The later described
		Widget Resource Editor offers this facility. But to define
		resources that should always be public, for all your project,
		use the dialog {\em Default Public}. You could add and delete
		resource names in the list.

	\myitem{\bf Help Menu}\index{Online Help}\index{Help System}The last
	menu in main window's menu bar is the {\em Help} menu. It serves as
	an entry point to help system.

	\myitem{On Help}
		The help system is opened and a text to describe the help
		system is displayed. If you are using the online help for the 
		first time you are encouraged to read this text.
		
	\myitem{Tutorial}
		The menu item Tutorial opens the tutorial on using the VDX.

	\myitem{On Context}
		By choosing On Context the mouse cursor changes and you may
		click on a dialog item you want help on.

	\myitem{About}
		The About menu item displays a dialog which contains VDX 
		Copyright information. If you licensed the VDX it will also
		display information about that license.

	% Toolbar
	\myitem{\bf Toolbar}
	To speed up certain often performed tasks, there are a few icons
	displayed below the Menu bar. By clicking on them you can easily
	take one of the following actions:

	With this icon you can load an existing project
	\psmargin{OpenProject.eps} file. You will be displayed with a file
	selection dialog in which you may choose a {\tt .prj} file to load.
	The action taken is the same as by choosing the menu item {\em Open}
	from the file menu.

	This icon is a shortcut \psmargin{SaveProject.eps} for the menu 
	item {\em Save} which saves the current project. If you save a new
	project for the first time a file selection box will appear in which
	you may choose a name for the project.

	This icon loads and displays the main template folder
	\psmargin{Templates.eps} as chosen in {\em Generic Options}.
	This folder allows you to use and edit the templates.

	By choosing this\psmargin{WidgetPalette.eps} icon the widget class
	palette will pop up. This palette shows all widget classes known to
	and usable by the VDX. The action taken is the same as with the menu 
	item {\em Palette} from the Widget menu..

	To create a new project,\psmargin{Generate.eps} besides the Project 
	menus {\em Generate} function you may also click this icon.

	To build and execute the current project including the build steps 
	xmkmf, make\psmargin{Execute.eps} you may choose either this icon 
	or the menu item {\em Execute} from the Project menu.

	\myitem{Test Mode} The {\em Test Mode} button\index{Test Mode} is
	used to switch the WYSIWYG display from layout to test mode.
	In layout mode you change the appearance of the interface by
	interactively moving widgets and changing their resources. 
	In test mode you can preview the behavior of the generated interface.

	\psmargin{Logo_16.eps} The icon showing the VDX logo may be used to
	display the About Box. It's the same as choosing {\em About} from the
	Help menu.

	% Interface IconView

	\myitem{\bf Icon View}\index{Icon View!Main Window}
	The main area of the VDX window is occupied by the Icon View.
	This view displays all interfaces in the current project.
	Within this view you may select objects by clicking on them with
	the left mouse button. By holding down the control key while clicking
	with the mouse on non selected objects these will be added to the
	current selection. If you select an icon and hold down the mouse
	button while dragging it over to other objects, these objects will be
	added to the current selection. Another way to select icons in the
	icon view is to drag the mouse cursor to an empty area in the view
	space and pressing and holding the left button while pulling with
	the mouse. You will see a so called rubber band forming a rectangle
	in which all enclosed objects will be selected.

	The name of an \index{Interface!changing name} interface is determined
	by the name of the first widget (Shell Widget) in that interface.
	Thus, by changing the name of the interface, the name of the first
	widget will be changed as well.
	This can be accomplished by pressing the Alt key and clicking on
	an interface with the left button of the mouse simultaneously. A
	input dialog will pop up with which you may change the interfaces
	name. The changes will take effect if you click the mouse outside
	the dialog or by pressing the \showkey{Enter} key. If you wish to
	abandon the changes made press the \showkey{Esc} key while the dialog
	is still active.

	\index{Interface!Kontextmenu}
	By clicking the right mouse button in the icon view a context menu
	will be displayed. If done while the mouse cursor is on a selected
	object, the context menu will apply to all selected objects. By
	choosing one of the displayed menu items i.e. {\em Browse}, browsers
	for all of the selected objects will be opened. Some of the context
	menu's items have key shortcuts such as \showkey{B} for browse which
	can be used directly on  without activating the	context menu with 
	the mouse. All these actions of course only work
	if the keyboard focus is within the icon view, which will be
	indicated by a black border.

	The following actions can be taken via the context menu of the
	interfaces:

	\myitem{Browse}\index{Browse}\index{Tree View!opening}This 
		item opens the browser. The key shortcut for this is
		\showkey{B}.
		The browser displays the widget tree for the selected
		interface. Each widget in the tree is displayed as an icon,
		which has the same behavior as an icon in the icon view.
		Additionally the browser allows changing the tree structure
		by using the clipboard and Motif's Drag \& Drop mechanism.


	\myitem{Save}\index{Interface!saving}The 
		save function allows the user to save a particular
		interface to a separate file. This file will have the
		extension {\tt .i}.
	
	\myitem{Save As}
		To save the interface under a different name select
		this menu item.

	\myitem{Create Link}\index{Link!creating}Creates a link to
		another interface. If the user wishes to reuse the
		interface in another widget he may create a link.
		To accomplish this first select this menu item, then
		point the mouse to the new parent widget within the WYSIWYG
		view of the interface the link should point to.

	\myitem{Generate}\index{Source Code!generating}Generate source code
		for a single interface. This is useful if you only changed
		one interface and don't want to go trough all the steps of
		rebuilding the whole source as with the Project menu's 
		{\em Generate}. There will be no stub files rebuilt as well.

	\myitem{Show}\index{WYSIWYG View}This 
		item is used to show the WYSIWYG view of the selected
		interface. The key shortcut is \showkey{S}.
		The WYSIWYG view allows the user to view and change the
		appearance of the interface. It facilitates moving and
		resizing of the interface's widgets.

	\myitem{Hide}Hides the WYSIWYG view. (Key \showkey{H}).

	\myitem{Delete}\index{Delete!Interface 
	context menu}\index{Interface!deleting}To 
		remove an interface from the project. This action does not
		delete the interface file.

	The WYSIWYG view will also be activated if an icon in the
	icon view will be double-clicked. {\em Show} is the so called default
	action.

	\myitem{\bf Output}\index{Output}The 
	window Output contained in the VDX main window will show
	messages normally written to standard output or standard error
	output. These are in particular warnings of Xlib, X Toolkit,
	Motif and informational messages of the View Designer. The
	{\em Clear} button allows to erase all messages in this window.

	\index{Quick Help}The 
	lower border of the main window contains the Quick help
	line. If the user moves the mouse over almost any item in
	the main window a short help text describing the function
	of the item will be displayed here. This help line is available 
	in nearly any of the View Designers windows. It can be disabled
	with the Generic options dialog.

\section{WYSIWYG View}
	
	\index{WYSIWYG View}
	The main purpose of an interface builder is the creation
	of graphical user interface objects. The only way to do
	this effectively is to show the result (the interface)
	while the user is editing it.

	\myitem{\bf Selecting}\index{Widget!selecting}Figure
	\ref{wysiwyg.eps} on page \pageref{wysiwyg.eps} shows 
	the WYSIWYG view of an
	example user interface. A button with the label {\em Cancel} is
	selected. The selection of a widget is done by clicking on
	it with the left button of the mouse. As with the icons in
	the icon view multiple selections are done by holding
	the Control key while left-clicking.
	
	\psfigure{wysiwyg.eps}{WYSIWYG view of an example interface}

	\myitem{\bf Resize}\index{Grid}\index{Widget!changing size}The 
	left mouse button is also used to change the size of a
	widget. To do so move the mouse cursor over one of the
	eight little rectangles on the border of the widget. Press
	and hold the left mouse button and drag the mouse in the desired
	direction to resize the widget. While dragging the mouse the new
	size of the widget will be shown as a black bordered rectangle.
	When releasing the mouse button	the widget will resize itself to 
	the new dimensions. The new size of the widget will be adjusted
	according to the {\em Grid} set in the Options menu of the main
	window.

	\myitem{\bf Move}\index{Widget!moving}\index{Drag
		\& Drop}The View Designer uses the Drag \& Drop mechanism
	of the Motif toolkit. This mechanism is common to all Motif
	applications and uses the middle mouse button.
	This makes it possible to move a widget around in its parent
	but also to move a widget from one parent to another or even
	to another interface. While doing this the WYSIWYG view will
	display a rectangle the size of the moved widget. According to
	the grid this rectangle will be moved relatively to the new
	parent widget, so the new position will always be visible.

	\myitem{\bf Copy}\index{Widget!copying}\index{Drag \& Drop}Drag 
	\& Drop is also used to copy certain objects. To create
	a copy hold the Control key during the whole process. After
	releasing the middle mouse button, the object will be copied
	to its new location, leaving the source unchanged. The newly
	created object may be assigned a name different from the sources
	because names must be unique within a single interface.
	
	\myitem{\bf Context Menu}\index{Widget!context menu}
	The right mouse button is used to open the objects context
	menu. If there is more than one widget selected, the menu
	will apply to all selected widgets.
	The context menu contains the following entrys:

	\myitem{Edit}\index{Edit}\index{Widget Resource 
		Editor!opening}By selecting Edit from the menu or by
		pressing \showkey{E} the widget resource editor will
		be displayed. This editor allows for the modification
		of a widgets resources. It can also be opened by double
		clicking on a widget with the left mouse button, again
		forming the default action.

	\myitem{Browse}\index{Tree View!opening}As with the menu item
		Browse in the icon view's context menu selecting this
		item opens the browser. Contrary to the above mentioned
		this browser will not contain the whole widget hierarchy
		of the interface but only the selected widget and 
		its children.

	\myitem{Align}\index{Align}\index{Widget!aligning}
		The sub menu Align contains eight menu items. These items
		will be activated when more than one widget is selected.
		It is significant which widget is selected first since
		the alignment will be adjusted according to this widget.

		\begin{mylist}{Horizontal Centers}
		\item[Left] All selected widgets will be aligned at the
			left edge of the first widget.

		\item[Right] The widgets will be aligned at the right
			edge of the first widget.
 
		\item[Top] All widgets align at the upper edge of the
			first widget.

		\item[Bottom] The selected widgets are aligned at the lower
			edge of the first.

		\item[Horizontal Center] All but the first widget will be
			moved horizontally until they center with the first
			widget.

		\item[Vertical Center] The centering will be done by
			moving the widgets vertically until they are
			centered with the first widget.
	
		\item[Width] All widgets inherit the width of the first
			selected widget.

		\item[Height] The widgets inherit the height.
	
		\end{mylist}

	\myitem{Cut}\index{Cut}\index{Clipboard}The Cut menu item causes
		the widget to be copied to the Clipboard and then 
		removed from the interface. The key shortcut for this action
		is \showkey{X}. If you selected multiple widgets all
		will be copied and removed. The cut widgets can be
		inserted into any interface by selecting the Paste
		menu item.

	\myitem{Copy}\index{Copy}\index{Clipboard}
		In opposite to Cut this item does not remove the widget
		from the interface it was selected in. The Paste action
		will as with Cut insert the copied widgets into an interface.

	\myitem{Paste}\index{Paste}\index{Clipboard}With the Paste entry
		widgets that have been Cut or Copied with the respective
		actions will be pasted into an interface. The widget whose
		context menu was opened to do the paste will become the
		new parent widget. If the widget can't become parent to
		the widgets in the clipboard the Paste menu item will
		be grayed and not selectable. {\em Paste} is also reachable
		with the shortcut \showkey{V}.

	\myitem{Menu Editor}\index{Menu 
		Editor!opening}\index{Menu Editor} This menu item is
		visible only for menu bar, pop-up menu and pull-down
		menu widgets (Key \showkey{M}). A new window will open
		which allows the creation of menu structures. With this
		Editor you will be able to effectively create menus
		with push buttons, toggle buttons, labels, separators and
		sub menus.

	\myitem{Visibility}\index{Visibility}\index{Dialog} As with the
		Edit menu entry this item will only be visible for
		dialog widgets. With it a dialog can be shown or hidden.
		Dialogs are parts of the interface that are visible only
		for a certain action, i.e. opening a file will show a
		file selection dialog which will disappear after a file
		is selected.

	\myitem{Refresh}\index{Refresh}\index{WYSIWYG View}
		The WYSIWYG view can be refreshed with this menu item.
		The shortcut key for this action is \showkey{F}.
		Refreshing the view may be necessary if some of
		the view's contents gets messed up due to errors while
		drawing.

	\myitem{Delete}\index{Delete!in Widget Context Menu}\index{Widget!deleting}
		The widget will be completely erased. If there are
		children contained within it you will be asked if you
		really want to remove the widget and all of its children.
		Attention! Contrary to the Cut action this cannot be
		reversed since no copy of the erased widgets will be
		saved to the clipboard. The shortcut for Delete is
		\showkey {Del}.

\section{Widget Resource Editor}
		
	\index{Widget Resource Editor}
	\index{Resources!changing}
	With the widget resource editor (see fig. \ref{resourceedit.eps})
	the resources of widgets can be changed.
	It can be opened by selecting Edit from the context menu or
	double clicking on the widget.

	\myitem{\bf File}\index{File!in Widget Resource Editor} In the
	file menu of the resource editor various file operations on a widget
	can be selected. It contains functionality to save 
	selected widgets to a file (item {\em Save} or key \showkey{Ctrl-L}),
	to load widgets from the hard disk (item {\em Load} or key
	\showkey{Ctrl-L}), to apply changed resources to the widget
	(item {\em Apply} or key \showkey{Ctrl-A}) and to close the resource
	editor (item {\em Close} or key \showkey{Ctrl-C}).

	\myitem{\bf View}\index{Category}\index{View!in Widget Resource Editor}
	The view menu of the resource editor allows selecting the
	the different views on a widget. The currently active view is shown
	in the line below the menu bar. The keys \showkey{Ctrl-1} to
	\showkey{Ctrl-5} switch between the different views.

	\myitem{Core}\index{Core}
		Within this category are contained all resources of the
		core widget class. This class is the base class for all
		other widgets, it contains resources common to all widgets.
		These are for example {\tt XmNx, XmNy, XmNwidth, XmNheight}.

	\myitem{Specific}\index{Specific}This category contains resources
		specific to the current widget class (e.g. XmNLabelString).

	\myitem{Constraint}\index{Attachments}\index{Constraint}The
		constraint resources for a widget are defined by it's
		base class. These resources are only necessary for
		specific widgets.
		If the parent widget defines the child widget's appearance
		and needs additional data for every child to do so, this
		data usually is contained within the constraint resource
		category, e.g. the children of a {\tt XmForm} widget have
		resources defining their preferred layout (Attachment).

	\myitem{Callbacks}\index{Callback}This category contains all
		callback resources of a widget. Callbacks are functions
		which will be executed if certain events occur.
		i.e. A certain callback routine will be called if the
		user clicks on a push button with the mouse. Callbacks
		therefore contain the main functionality of your
		application. The View Designer cannot help you
		in designing this functionality, what it does is to
		help you layout your applications front end.

	\myitem{Extra}\index{Extra}\index{Widget!Name}Special adjustments
		which are not in particular resources can be made with
		the entrys in this category. These can be the name for
		the widget, or the toggle button {\em Manage} which
		assures that the widget will be made visible after creating it.

	% Bild

	\psfigure{resourceedit.eps}{Widget Resource Editor}

	
	\myitem{\bf Automatic Load}\index{Automatic Load}The standard
	behavior of the View Designer is to open a new resource editor
	each time {\em Edit} is chosen from the context menu. The toggle
	button {\em Automatic Load} in the options menu switches it
	to loading the resources of further edited widgets into a single
	specific resource editor. If you choose to open a resource editor
	for a widget after enabling this button no new resource editor
	will pop up but the resources will be loaded into the editor
	specified after selecting {\em Automatic Load}. The status of
	automatic load can be viewed in the status line below the menu
	bar.
	
	\myitem{\bf Resource lines}\index{Resource lines}
	The main area of the widget resource editor is occupied by a
	scrollable window containing the resource lines available
	for the particular widget. On the left hand of each line is
	a toggle button showing the name of the resource. If this button
	is active, the selected resource is marked for manual setting
	in the source code or the resource file. If the button is not
	active the default value for this resource will be used.
	On the right side of the button various fields may be shown, depending
	on the resource type. These can be e.g. input areas, option menus
	or toggle buttons. If the resource is for example boolean,
	a toggle button will be shown, which allows for setting the
	value of the resource to true or false. The next field to the
	right contains push buttons through which a text editor,
	a font chooser or a color selection dialog may be opened. The
	options menu on the rightmost side of the scrolled window allows
	the user to choose a source for the resource values. The
	following sources can be chosen here:

	\myitem{private}\index{private}\index{Source Code}
			The resource value will be written to the created
			source. The View Designer separates the generated
			source into two different kinds of source files.
			The files edit-able by the user, called Stubs and
			the files used by the View Designer internally
			which must not be edited since these files will
			be overwritten each time source code is generated.
			One of the latter files also contains these private
			resources.

	\myitem{public}\index{Resource file}\index{public}This resource
			will be contained in an external file created while
			generating the source code. The source created will
			contain a file with the extension {\tt .rf}. Building
			the application a file with the name 
			{\em project name}{\tt .ad} will be created containing
			all the resources from these {\tt .rf} files.

	\myitem{load}\index{load}
			For resources of the Pixmap type
			\index{Xpm file}\index{Pixmap}this menu item
			specifies the pixmap file (Xpm format) to be 
			loaded at run time.
			This means the pixmap file will be loaded after the
			application has been started.
			The text field of this resource must contain
			the path and name of the pixmap file. The path may
			be absolute or relative or may just contain the
			name of the pixmap file in which case the file
			will be looked up with {\tt XtResolvePathname()}
			\index{XtResolvePathname()} which looks for a
			subdirectory {\tt pixmaps}. This will affect both
			the created application and the WYSIWYG view.

	\myitem{include}\index{include}Xpm files can be included in the
			generated source. If your application shouldn't
			bother with looking for {\em possibly missing} pixmaps
			it could be useful to compile the pixmaps into
			your applications executable. The Xpm file will
			be included with the {\tt \#include} C preprocessor
			statement.

	\myitem{method} If \index{method}\index{Callback}option
			menu shows this entry, the name of a callback
			function could be entered here. This function
			will be located in the above mentioned Stub files
			generated by the View Designer.
			The functions code will have to be implemented
			by the developer of the application, respectively
			you, the user. The test mode of the WYSIWYG view
			is designed to help you in finding out if your
			callback will be branched to at the desired action.
			If you play with the interface in test mode, the
			lower part of the main window (message pane) will
			show information which widget called one of its 
			callbacks.

	\myitem{manage} In this input field the name of a widget should
			be entered\index{manage}\index{Dialog}.
			The widget which corresponds to the name 
			(e.g. a Dialog) should be in the same interface and
			will be shown in case this callback occurs.
			This is especially useful for dialogs, 'cause
			a dialog is usually invisible and is only shown
			if a certain action on the interface is carried out
			e.g. if a File open menu item is selected in which
			case a file selection box is shown.

	
	There is a Quick help bar giving a fast overview of all the
	elements in the resource editor too.

	The push button {\em Apply} on the lower border will apply
	the changed resources to the widget.
	{\em Close} will close the window discarding all changes made
	since the last {\em Apply} command.
	The {\em Help} button will take you to the online help.

\section{Browser}

	\index{Tree View}
	As shown in figure \ref{browser.eps}, the View Designer incorporates
	a Tree View. It provides for a quick overview of the widget hierarchy
	of an interface or a widgets sub tree. The browser is opened
	by choosing {\em Browse} from a widgets context menu. The plus and
	minus signs in the tree view open or hide parts of the tree. The
	icon representing the widget shows information of what type the widget
	actually is.

	\myitem{\bf File}\index{File!in Browser}The file menu of this window
		contains the file operations available for the widget
		shown in the browsers window.

	\myitem{Load}\index{Load!in Browser}This menu item - as well as the
		key \showkey{Ctrl-L} - loads a widget from the disk into
		the browser. The widget to be replaced may be chosen from
		the WYSIWYG view or from any icon or three view. If an
		interface is chosen in the main window, the shell widget
		of that interface will be loaded.

	\myitem{Load Parent}\index{Load Parent!in Browser}The parent
		of the currently displayed widget will be loaded with 
		{\em Load Parent} or the key \showkey{Ctrl-P}. If there
		already is a shell widget displayed in the browser this
		action has no effect.

	\myitem{Close}\index{Close!in Browser}Closes the browser window.
		The key shortcut for this is \showkey{Ctrl-C}.

	\myitem{\bf View}\index{View!in Browser}The {\em View} item allows
		you to change the properties of the browser window, e.g.
		displaying large or small icons. The defaults for these
		settings can be changed with the View Defaults dialog of
		the View Designer's main window. Note that the changes you
		make are only applicable for the particular browser you
		are editing. To change the overall behavior of all browser
		windows use the View Defaults dialog.

	\myitem{Small Icons}\index{Small Icons!in Browser}If this menu
		item is chosen, the tree view will contain only small
		icons for each widget. It can be used to fit more widgets
		into the browser's window.
	
	\myitem{With Icons}\index{With Icons!in Browser}If you don't want
		to see any icons, choose this item to switch them on or off.

	\myitem{With Class Names}\index{With Class Names!in
		Browser}The default browser shows the name for each widget
		class displayed in its window. The names will be shown to
		the right of the icons in italics. This menu item toggles
		displaying these names.

	\myitem{\bf Widget}\index{Widget!Menu in Browser}This 
	menu contains actions also available in the context menu of
	each widget. The actions apply to all widgets selected in the tree
	view. If no widget is currently selected these menu items will
	be grayed and can't be selected.

	\myitem{\bf Options}\index{Automatic Load!in Browser}The options menu
	currently contains only one entry. The toggle button
	{\em Automatic Load} does the same as the button in the widget
	resource editor. The currently active browser can be chosen to
	display all tree views opened (context menu {\em Browse}) after 
	toggling this button to on.

	\psfigure{browser.eps}{The Browser}

	\myitem{\bf Drag \& Drop}\index{Drag \& Drop}In addition to the 
	actions offered by the icon view the browser allows for moving
	widgets within the tree. This is again accomplished using the
	Motif Drag \& Drop feature. By dragging the mouse while holding
	the middle button down the selected objects can be moved to
	a different position within the tree. There are two possible
	actions that depend on the position of the mouse pointer within
	the tree while doing the Drag \& Drop. These actions can be
	determined by the shape of the mouse cursor shown while dragging.

	\begin{enumerate}
	\item Insert as a child. The mouse cursor will show an angle. The
		selected object will be inserted as a child of the object
		the mouse cursor is currently on.
	\item Insert in between. The mouse cursor will show no angle. The
		selected object will be inserted between two neighboring
		objects. The mouse cursor must be positioned between those
		two objects to carry out this action.
	\end{enumerate}
	
	Similar to the WYSIWYG view this view allows to copy instead of
	moving the selected objects by pressing the Ctrl key while
	dragging.

	It should be noted that not every widget is allowed to have children.
	For example a push button never can have a push button child.
	If the user chooses a wrong Drag \& Drop action the failure
	will be indicated by the mouse cursor being dragged back to where
	the invalid move began.

	The other options available with the tree view are the same as
	for the icon view. One or more objects may be selected and a pop
	up menu for them may be opened. The default action, activated by 
	double clicking with the left mouse button, is opening the
	widget resource editor. Changing the name of a widget is also
	possible. This is done by holding the \showkey{Alt} key while
	clicking with the left button on the desired widget, this provides
	a much more efficient way to change widget names than the
	resource editor does.

\section{Menu Editor}
		
	\index{Menu Editor}
	To allow the user to easily create menus, the View Designer has
	a built in menu editor (Fig. \ref{menuedit.eps}) function. This menu
	editor can be activated via a widgets context menu. The menu
	editor feature will only be available for widgets to which applying
	a menu is possible. These are e.g. menu bars, pop-up menus, pull down
	menus or radio boxes.

	The menu bar of the menu editor provides the same functionality as
	the menu of the browser. The file menu allows you to load a widget
	into the menu editor or to close the menu editor. The view menu
	controls the icon view. The widget menu allows to perform actions
	on selected widgets in the icon view and the options menu contains
	the toggle button {\em Automatic Load}. The default appearance of
	the menu editor can also be changed with the View Defaults dialog
	of the main window.
	
	\psfigure{menuedit.eps}{Menu Editor}

	\index{Icon View!Menu Editor}The 
	icon view of the menu editor shows all children of the currently
	edited widget. It provides all the above mentioned functionality
	of an icon view. The figure shows a push button {\em Exit} which
	is currently being renamed.

	\myitem{\bf Folder}\index{Folder}The arrows are used to change the
	order of the child widgets. By pressing the desired arrow button
	the selected widget will move up and down the hierarchy.

	\myitem{\bf Add item}\index{Add item}The buttons before {\em Add item}
	are used to create widgets. The {\ Submenu} item is special, because
	with this button a cascade button and a pull-down menu will be
	created. With the cascade button the resource
	{\tt \index{XmNsubMenuId} XmNsubMenuId} will be set, creating
	a motif sub menu.
	The created widgets will be inserted as last children and may be
	edited later. For example by pressing \showkey{Alt} and the left
	mouse button while the cursor is over an icon, the name of the
	corresponding widget may be changed. Double clicking on an icon
	opens the widget resource editor and right clicking on it opens
	the context menu for the widget. If you've created a sub-menu, the
	context menu will contain an item menu editor which will open another
	menu editor for this particular sub menu. This way, menu
	structures of any depth can be edited.

\section{Project Settings}

	\psfigure{settings.eps}{Project settings}

	\index{Project Settings}\index{Settings}
	The project settings can be reached by choosing {\em Settings} in the
	project menu. A window as shown in figure \ref{settings.eps} will
	be opened. In this window settings mostly regarding source code
	generation may be altered.

	\myitem{\bf Project File} The text field {\em Project file} contains
	the name of the projects main file. \index{Project File}
	\index{Application Class}\index{Executable} This file
	will have the extension {\tt .prj}. The extension may be left out when
	entering a new name, since it will be added by the View Designer if
	it's missing. If you change the project file name and use \showkey{Tab}
	to switch to another input field, the {\em Application class} and
	{\em Executable} will be changed according to the project name.
	These presets may be changed at will, they are only suggestions.
	It may be noted, however that the executable name should be lowercase
	and the application class name should start with an uppercase letter.
	The application class sets the name for the later created resource
	file.

	\myitem{\bf Generation Model} The type of source code created is
	set by the menu {\em Generation Model}. The View Designer contains
	two language models at the moment, C and C++. The setting made here
	will affect the Imakefile, the main program and the created user
	interfaces. How to manipulated code generation can be seen on page
	\pageref{codegeneration}. In the main program a certain interface
	will be created and displayed first \index{Startup Interface}.
	Which interface should be first can be adjusted
	with the menu {\em Startup Interface}.

	\myitem{\bf Source Path}\index{Source Path}The {\em Source path}
	setting determines where the source for the project should be
	generated. The default will be the same directory the project files
	are stored in. The input field {\em Source path} may contain an
	alternate path to store the source. For example you may enter
	{\tt src} in which case the source will be stored in a subdirectory
	{\tt src} in your project directory. This will separate the
	generated source from the project files. If the entered directory
	does not exist an attempt will be made to create it. The small button
	close to the input field opens a file selection dialog which may
	be used to select the path.
	
	\myitem{\bf More Source Files}\index{More Source Files}Since your
	program will consist of more than only the View Designer created
	source the {\em More Source files} input field may contain additional
	source files. The names must be separated by spaces. The button at
	the input field again opens a file selection box in which all
	C and C++ sources will be listed. You may now choose the files
	to be included in the generated source and they will be entered
	into the input field after closing the selection box.

	\myitem{\bf Extra ..}\index{Extra Compiler Options}Additional flags
	for the Compiler and Linker \index{Extra Linker Options} may be entered
	in the following input fields. A compiler option could be e.g. 
	{\tt -I/opt/\-dbms\-/include} which will search for include files
	in the given extra path besides the standard include path.
	A linker option would be e.g.  {\tt -L/opt/dbms/lib} which would add
	{\tt /opt/dbms/lib} to the linker search path, and 
	\index{Extra Libraries} extra libraries could be {\tt -lsql}.
	These are only examples, the options valid to your compiler may
	differ.

\section{Templates}

	The template window (figure \ref{template_window.eps}) shows the
	template file (extension {\tt .pal}. This file holds widgets
	and widget trees used as templates. The main template window
	can be opened by an icon in the View Designer's main window.

	\psfigure{template_window.eps}{Template Window}

	The icon view displays the contents of the template file. This view
	is special from the other icon views because if the user clicks on an
	icon in this view, the mouse cursor changes and the View Designer
	waits for the user to point it to where a new copy of the template
	should be created. If you click in an WYSIWYG view a widget will
	be created within the view which has the same properties as the
	template widget. If you click outside the WYSIWYG views the View
	Designer will create a new interface using the template. The right
	mouse button will cancel the action.

	The menu of the template window is almost the same as in the 
	widget browser and the menu editor. The view menu and the widget
	menu are in fact identical to those. Therefore the view menu,
	as in the other views, controls the behavior of the icon view.
	The widget menu contains the same items as in the context menu
	of each widget (The special Paste to templates is discussed a
	little later).
	
	\myitem{\bf File}Somewhat different is however the file menu.
	It allows for saving of the modified templates ({\em Save} or
	\showkey{Ctrl-S}). As usual a file selection box will appear and
	you may choose a different filename for the template file. {\em Close}
	or \showkey{Ctrl-C} closes the window. If the templates have been
	changed a message box will appear asking you whether to abandon
	the changes or not.

	\myitem{\bf Paste to Templates}\index{Paste!to Template}
	The item special to the template view's widget menu is the item
	paste to templates. With this item the contents of the clipboard
	will be copied to the template window. This option is not tied
	to the widget selected in the window. It may be chosen only
	if one or more widgets are held in the clipboard.

	\myitem{\bf Create On Click}\index{Create On Click}The 
	menu item Create on Click, contained in the option menu, changes
	the earlier mentioned behavior when clicking on an �con in the icon
	view. If this option is toggled off, icons may be selected with
	the left mouse button. The Create on click option along with
	the window size and the size of the icons is saved for your
	convenience.

	\myitem{\bf Drag \& Drop}\index{Drag \& Drop!in Templates}
	The Drag \& Drop mechanism of course is also available with the
	template view. The user can drop widgets into the window and
	they will be inserted into the templates. The reverse of this action
	is also possible so that the dragged template can be copied to an
	existing interface by dropping it there.

\chapter{Sample Application}

	The application design process using the View Designer will be 
	shown in this chapter with the Textedit application.

	First, create a directory for the application. This directory
	will contain the files used by the View Designer as well as
	the source code of our sample application. Change to the directory
	you just created and start the View Designer.

\section{Project Settings}

	\index{Project Settings}
	After starting the View Designer, a project called {\em Untitled}
	is already created. To change the project name, use the {\em Project}
	menu item {\em Settings}. The dialog already shown in figure
	\ref{settings.eps} will be displayed.

	Our project name shall be Textedit. Enter 
	{\tt Textedit.prj} into the first input field. You may omit the
	extension {\tt .prj} for the View Designer will append it
	automatically in that case. The file {\tt Textedit.prj} is
	the main project file and contains information about the
	project. Since we want our project to be C++, select the
	corresponding option in the dialog. Now select {\em OK} to
	close the window and apply the changes to the project.

\section{Creating an Interface}
	
	An interface will be added implicitly if you create a widget
	without creating its parent first. If this widget is a shell
	this shell will become the first and only widget of the interface.
	If the created widget is no shell, the necessary shell will be
	created automatically. The shell widget will be a default
	shell in that case, which can be selected in the options menu
	of the main window. An exception to this is creating a dialog.
	A dialog can't be created without a parent widget, since dialogs
	can only be parts of an interface.
	
	A widget can be created in two different ways.
	\begin{enumerate} \index{Widget!creating}
	\item 	The menu {\em Widget} of the main View Designer window
		is used. It contains the sub menus {\em Shells}, {\em Manager},
		{\em Primitives} and {\em Dialogs}. These sub menus
		contain all widget classes known to the View Designer.
		The sub menus are tear off menus and therefore can be
		pinned to a separate window on the desktop using the
		uppermost dotted menu entry.
	\item	The widget class palette is opened using the menu item
		{\em Palette} in the {\em Widget} menu or the palette
		icon in the main windows tool bar. This palette is
		split into the same categories as the {\em Widgets} menu.

		The icon views displayed in the palette show the different
		widget classes, but to save space without names. The name
		of the widget class under the mouse cursor is shown in
		the palette view's quick help line.

	\end{enumerate}
	When a widget to be created is selected as shown above, the mouse
	cursor will change indicating that the View Designer expects the
	user to select a position for the new interface. If the left
	mouse button is pressed while dragging with the mouse a rectangle
	will appear indicating the size of the widget to be created.
	With releasing the mouse button the widget will be created.
	It should be noted however that a new interface will only
	be created if the position the mouse is clicked isn't in a
	WYSIWYG view. 
	Otherwise an attempt will be made to create the new widget
	within the existing interface.

	\index{Default Shell}
	For our sample application we will create a widget of the
	XmMainWindow first. Because the default shell is TopLevelShell
	\footnote{ In the later created source an Interface will be
	represented by a pop-up shell. The ApplicationShell
	will always be created, therefore it isn't necessary to
	create it explicitly. } we don't have to worry about the
	shell widget. To do so select {\em MainWindow} in the {\em Manager}
	sub menu and create a widget as described above. A widget of
	type TopLevelShell named {\em Interface\_1} and a widget
	named {\em mainWindow\_1} will be created.
	The icon view of the main window will now contain an icon for
	the newly created interface.

	Now our application needs a menu bar. We can create it by using
	the menu item {\em MenuBar} in the {\em Manager} sub menu. Note
	that unlike XmMainWindow {\tt XmCreateMenuBar()} is no widget class
	but a Convenience function. To create the menu bar, this function will
	be called in the generated source. To create XmMainWindow the function
	{\tt XtCreateWidget()} will be called.
	
	When you selected the menu entry {\em MenuBar} don't click the mouse
	on an empty area of the desktop, since that would create a new 
	interface containing the menu. Instead click and drag in the
	WYSIWYG view of our first interface. Note that the regardless of
	the size you dragged the widget will always be on the upper edge of
	the interface.

	\psfigure{interface_1.eps}{The new Interface}

	\index{WYSIWYG View}
	Figure \ref{interface_1.eps} shows the WYSIWYG view of the new
	interface.

\section{Using the Menu Editor}
	
	\psfigure{menuedit_bar.eps}{Menu Editor}

	\index{Menu Editor!using}
	Well, doesn't the menu look a little empty? Press the right
	mouse button on the menu bar and select {\em Menu Editor} from
	the context menu. The menu editor as shown in
	figure \ref{menuedit_bar.eps} will open up. However the icon
	view as opposite to the figure will be blank.

	\psfigure{browser_bar.eps}{The Menu Hierarchy}

	A menu bar only can hold sub menus, therefore only the push button
	{\em SubMenu} will be selectable. Press it twice now to create
	the following four widgets: two cascade buttons and two pull-down menus.
	Each pull-down menu is connected to one cascade button, i.e. the
	resource {\tt XmNsubMenuId} contains the name of the pull-down menus.
	The names for the new widgets should be changed to be more verbose. To
	do so use \showkey{Alt-Left Mouse Button}. The renaming
	procedure is shown in figure \ref{menuedit_bar.eps}. If the name
	for a pull-down menu is changed, the corresponding {\tt XmNsubMenuId}
	of the cascade button will be changed as well.

	\index{Icon view!Menu Editor}
	The icon view also allows for opening context menus for the 
	widgets. To open a menu editor for the sub-menu {\em FileMenu}
	simply choose {\em Menu Editor} from the context menu. This way
	the sub menus can be filled with items building a widget hierarchy
	as shown in \ref{browser_bar.eps}. 

\section{Using the Widget Resource Editor}

	\psfigure{resourceedit_new.eps}{Changing Resources}

	\index{Widget Resource Editor!use}
	Although the widget names are meaningful for the programmer now,
	the end user wouldn't know what to do with them. Therefore,
	every widget needs it's {\tt XmNlabelString} to be changed.
	To support internationalization, it would be useful to store
	the labels in a separate external resource file.

	Choose a menu item, e.g. the widget {\tt New\-Button} and open
	the the widget resource editor. Enable {\em Automatic load}.
	The category specific contains the {\tt XmNlabelString} resource.
	The resources are sorted alphabetically and are displayed without
	the preceding {\tt XmN}. If you change the text in the text field
	the toggle button indicating a manually set resource will be
	enabled. You can reverse this by disabling the button again.
	The asterisk in front of the resource indicates that this resource
	has been altered but not applied (using {\em Apply}) to the widget.
	The leftmost pop-up menu contains settings for using an external
	resource file {\em public} to set this resource or for setting
	the value for the resource within the source code {\em private}.
	We want the resource to be read from a resource file, so we choose
	{\em public}.

	After pressing the {\em Apply} button, all changes made will be put
	to the widget. Now we choose another widget e.g. {\tt OpenButton},
	without closing the resource editor first, by double clicking with
	the left button or choosing edit in the context menu for the button.
	The resources for the new widget will be shown in our
	resource editor. This is exactly the effect the {\em Automatic load}
	menu item has. If it isn't activated the time we choose the new
	widget a separate resource editor will be opened.

	\index{Callback}
	The widget resource editor is also used to define the callbacks for
	the widgets. The category {\em Callbacks} contains the callbacks
	valid for the particular widget. The text fields allow the user to
	enter the name of the function which is to be called if the callback
	occurs. The code for the callback is to be implemented by the
	application programmer (you) and is located in a separate file not
	normally touched by the View Designer. The editing of the source
	code is not done with the View Designer since there are enough
	excellent editors out there and each programmer has his own favorite.

	\psfigure{editwindow.eps}{The Edit Window}

	So far the only functionality our application has is a menu.
	The thing to add for a text editor is of course a text widget
	with which the text editing can be done. If the text widget is
	created as a child of {\tt XmMainWindow} it will fill up the
	entire client area of the window. Only the menu bar will be visible.
	Furthermore a scrollbar will be placed on the lower edge of the
	text widget. If we change the resource {\tt XmNeditMode} to
	XmMulti\_LINE\_EDIT the widget will change from a one line text
	field to a multi line editor.

	What the figure so far doesn't show are the file selection boxes
	for loading and saving the edited text and a MessageBox dialog
	which will be shown before the application terminates.
	
\section{Dialogs and Callbacks}

	\index{Callback}\index{Dialog}
	The procedure to create a dialog is similar to creating other
	widgets. If you click on a dialog widget class in the palette,
	the mouse cursor changes to a familiar shape. Click in the
	WYSIWYG view of the application interface to create the dialog.
	The context menu of the dialog has an entry labeled
	\index{Visibility} {\em Visibility}. With it you can hide the dialog
	in the View Designers context. Note that this does not change the
	visibility of the dialog for the created application. Every dialog
	defaults to create invisible (not managed) in the application
	source code. If you wish to pop up the dialog as soon as the
	application is started you will have to change this in the
	{\em Extra} category in resource editor.

	The View Designer allows the user to tie a dialog to a particular
	callback. If this callback occurs, it causes the dialog to
	become managed, which will make it visible. To do so enter
	the name of the dialog widget in the {\em Callback} category of
	the resource editor. Additionally the option pop-up on the
	left must be changed from {\em method} to {\em manage}.

	\index{Test Mode}
	The test mode allows you to check the correct binding of the
	callback to the dialog. This is particularly useful if the
	dialog should be opened by a push button or another visible
	part of the interface, e.g. the {\em Open} menu item.
	Our example ties dialogs to the menu items {\em Open}, {\em Save}
	and {\em Exit}.

\section{Source code generation}

	\index{Source code!generating}
	To generate the source code for the whole project, choose
	{\em Code generation} from the {\em Project} menu. A dialog
	will be displayed which allows you to choose what files the
	View Designer should create.

	\begin{enumerate}
	\item If {\em Main program} is chosen, the code for the main
		program will be generated.
	\item {\em Imakefile} will create a file that is used by {\tt xmkmf}
		to create makefiles for the application.
	\item The code for an interface is put in two files, one of them
		is overwritten by the View Designer, the other file contains
		the code written by the application programmer. It will
		contain the callbacks and will not be touched by the View
		Designer. However if it is the first time the source is
		generated it is useful to check the {\em Stubs} button
		since the View Designer will then create a file with empty
		callback stubs which can be easily edited by the programmer.
	\end{enumerate}

	After pressing {\em Ok} a list of the created files is shown for
	you to check whether an important file is to be overwritten. If you
	click {\em Write} the files will be finally generated.

	In our example the following files will be created:

	\myitem{\tt TextWin\-dow.C}
				This file contains the above mentioned
				callback stubs. The programmer must edit
				this code. In our case this file will
				contain the methods of a C++ class named
				{\tt TextWindow}. Also contained herein
				is the initialization of the module, which
				for us is the constructor of {\tt TextWindow}.

	\myitem{\tt TextWin\-dow.h} This header file contains the class
				definition of {\tt TextWindow}. New
				variables and methods should be added here.
				Remember that the initialization of the
				new members should be done in the constructor.

	\myitem{\tt TextWin\-dow.rf} All resources marked {\em public} in
				the resource editor will be written to
				this file. These {\tt .rf} files will be
				copied to one large resource file with
				the name {\tt Textedit.ad}. The {\tt .rf} files
				should never be edited.

	\myitem{\tt TextWin\-dow\_base.C} This file contains the base class
				definition {\tt TextWindow\_base}. It is the
				parent of {\tt TextWindow} and contains
				layout information for the interface. It also
				creates the widget tree for the interface.
				This file should never be edited.

	\myitem{\tt TextWin\-dow\_base.h} This is the header file for
				{\tt TextWindow\_base} containing the class
				definition. This file should never be edited.

	\myitem{\tt textedit.C} This file contains the {\tt main()} function
				which creates and opens the startup interface.

	\myitem{\tt VcView.C} {\tt VcView} Is the base class of all interfaces.
				(This is only valid for C++ code generation.
				{\tt VcView} contains functionality needed
				by every interface.

	\myitem{\tt VcView.h}	The header file of {\tt VcView}.

	\myitem{\tt Imakefile}  The Imakefile of the project. It is used by
				{\tt xmkmf} to create a makefile.
	
	To compile the application, first execute {\tt xmkmf}. A Makefile
	will be created. The next step is to {\tt make depend} to add
	dependencies to the makefile. With the final {\tt make} the code
	will be compiled and linked into an executable file.

	\index{xmkmf} \index{make}
	The above steps can also be done within the View Designer by choosing
	{\em Execute} from the {\em Project} menu.

	At the moment the application only has a dead user
	interface. To bring it to life you must implement the necessary
	callbacks. On how to implement the callbacks please refer to
	the example TextEdit shipped with the View Designer. It can be
	found in the directory {\tt doc/samples/TextEdit}. Please browse
	through {\tt TextWindow.h}, {\tt TextWindow.C} and {\tt textedit.C}.
	
\chapter{Additional Features}

	This chapter contains explanations for additional features of the
	View Designer which are there to ease working with the View Designer.

\section{Online Help}
	
	\index{Online Help} \index{Help System}
	The View Designer help system as well as the quick help are designed
	to be accessible from almost every dialog or editor to quicken the
	process of creating a complex application as much as possible.

	\myitem{\bf Help On Help}
	The help system can be opened e.g. by using the Help menu of the
	main window. The use of the help system is explained in Help On
	Help by choosing the appropriate button. The shown text explains
	the usage of the help system.

	\myitem{\bf On Context}\index{Context help}An online helps main
	purpose is to aid the user in all steps of the development
	process. If you want help on any window, choose {\em On Context}
	in the main windows help menu. The mouse cursor changes shape and
	you may click on the dialog item you want help on. The help system
	will search the Online help for an appropriate entry and will then
	display the corresponding help text.

	The context help is also available via the key shortcut \showkey{F1}.
	The keyboard focus must be on the dialog item you want help on.

	\myitem{\bf Tutorial}\index{Tutorial}Since you're already reading the
	manual, you have a basic understanding how the View Designer works.
	However the help system also contains an online manual. It describes
	the View Designers functionality in a similar way this manual does.

\section{Using Templates}

	\index{Template}
	Using templates is extremely useful when building large applications
	which contain many widgets of the same appearance. To achieve a
	uniform look and feel in your application, you can create widgets
	with predefined resources. This removes the necessity to edit
	the resources for each widget separately.

	A template doesn't need to consist of a single widget. Since widgets
	may have children the creation of template widget trees is also
	possible.

	\myitem{\bf Creating Templates}\index{Template!creating}After creating
		a widget you want to become a template you have to open
		the template window first. With the Drag \& Drop mechanism
		you then put this widget into the template window to make
		it a template. To do so click with the middle mouse button
		on the widget and drag it - while holding the middle button
		down - to the template window. The original widget will
		remain untouched, while the template window will show a
		new icon. This icon has a context menu similar to the original
		widgets menu. This way you can edit the resources of the
		template widget in the same manner as with the other widgets.
		Opening the widget browser and other actions are possible
		as well.

	\myitem{\bf Using Templates}\index{Template!using}Now the created
		template should be the base for a new widget. To do so
		left-click on the template icon. The mouse cursor will
		change its shape once again and the View Designer awaits
		you to choose a parent for the new widget. Click with the
		left button on the parent and the new widget will be
		created using the resources of the template.

	The mentioned procedure isn't the only way to create widgets
	from templates, you may also use Drag \& Drop or the Clipboard.

\section{Creating links to interfaces}

	\index{Link}
	The disadvantage of templates is that a widget, once created doesn't
	keep track of changed template resources. If you, for example,
	created a customer data form and used it thoroughly throughout your
	application and a change in the wanted data occurs, you will have
	to change every single form in your whole application. That
	could be a lot of work for sure.

	With links this disadvantage disappears. A link to an interface
	is created within other interfaces. If you change the linked
	interface, the change will affect all interfaces this widget was
	linked to. Even the functionality implemented in callbacks and
	other functions used in the original interface are the same
	for all interfaces it is linked to.

	\myitem{\bf Example}
		A customer data form is implemented as a separate interface.
		You wrote functions which read from and write to the form.
		If you use this interface as a link, this functionality is
		available to all interfaces that use this link. If the data
		sheet changes (e.g. a field is added) and the form changes
		these changes automatically affect all interfaces.

	Interfaces are therefore reusable components which can be bound
	to other interfaces using links.

	\myitem{\bf Create}\index{Link!creating}After creation of an interface
		as reusable item you use the context menus {\em Create Link}
		to start the linking. The mouse cursor changes shape awaiting
		the choice of a new parent widget for the link. The link will
		be created after clicking with the left button and will be
		displayed as frame widget with children. These children have
		the same hierarchy as the original interface. However the link
		will be treated as a whole and therefore is a frame widget
		in the hierarchy of the links destination (the new parent).
		The look of the interface can only be changed by editing
		the links source interface.

	\index{Interface}
	The fact that interfaces themselves are reusable components is not
	only important for the creation of links. It is also used with
	dynamically creating widget hierarchy. The modules created by
	the View Designer are also reusable there. The generated source
	code contains the interfaces as C structures or C++ classes.
	The function for initialization is contained in the file stub
	edited by the programmer and has the following form:

	\bigskip \hrule
	\begin{verbatim}
	InterfaceName *
	createInterfaceName(Widget parent,
	                    ArgList args, 
	                    Cardinal n);  

	\end{verbatim}
	\hrule \bigskip

	The {\tt parent} is the parent widget, {\tt args} is a resource
	list for the Xt Toolkit and {\tt n} specifies the size of the list.
	The {\tt InterfaceName} is to be replaced with the actual name
	of the interface to be created. The function creates an interface
	and returns a pointer to the created object. You may also change
	this function and pass a pointer to the data that should be
	viewed to the function. If however the interface is used as a link
	the function is defined as interface and should not be changed.

\section{Other Drag \& Drop features}

	\index{Drag \& Drop}
	You already heard of the use of the Drag \& Drop mechanism in
	many parts of the design process. The following list shows all
	possible actions that can be done by
	dragging and dropping.

	\myitem{\bf Widget creation}
	\begin{tabular}{p{3cm}@{\hspace{0.5cm}\arrow\hspace{0.5cm}}l}
	\hfill Widget class & WYSIWYG view	\\
	\hfill Widget class & Browser view \\
	\hfill Widget class & Templates \\
	\end{tabular}

	\myitem{\bf Widget movement}
	\begin{tabular}{p{3cm}@{\hspace{0.5cm}\arrow\hspace{0.5cm}}l}
	\hfill Widget & WYSIWYG view \\
	\hfill Widget & Browser view \\
	\end{tabular}

	\myitem{\bf Widget copy}
	\begin{tabular}{p{3cm}@{\hspace{0.5cm}\arrow\hspace{0.5cm}}l}
	\hfill Widget & WYSIWYG view \\
	\hfill Widget & Browser view \\
	\hfill Widget & templates\\
	\end{tabular}

 	\myitem{\bf Text export}
	\begin{tabular}{p{4cm}@{\hspace{0.5cm}\arrow\hspace{0.5cm}}l}
	\hfill Widget & Text entry field\\
	\hfill Widget class & Text entry field\\
	\hfill Interface & Text entry field\\
	\end{tabular}


\chapter{Configuring Code Generation}

	One advantage of the View Designers code generation mechanism is the
	use of template files. By using those templates, the programmer can
	suit the code generation process according to his needs. How to
	take advantage of template files will be shown in this chapter.

	\label{codegeneration}

\section{Code Generation Models}

	\index{Code generation model}\index{lang.cf}
	While generating the source code the View Designer reads its
	configuration files from {\tt lib/X11/vdx}. The file {\tt lang.cf}
	contains a list of all available languages.

	\index{Project options}
	As we already know the project settings dialog sets the language
	the source code should be generated in. But this does not only
	affect the generated source. The code generation module also
	determines which template files it should use from this setting.
	The View Designer ships with two built in code generation modules -
	{\em c} and {\em c++}. You may create your own code generation
	with the name of {\em my-c++} for example. If you did, the settings
	dialog will display a third model, which you created yourself.

	\myitem{\bf Model creation}
	\index{Code generation model!creating}\index{addLang} A new generation
	model is created using the shell script {\em addLang}. It is located
	in the same directory the template files for existing languages are
	in. To create a new model change to the directory {\tt lib/X11/vdx}.
	You'll have to use an existing model to create the new one. The
	syntax for {\em addLang} is as follows. 
	\begin{verbatim}
	addLang c++ my-c++
	\end{verbatim}
	This will copy the {\em c++} model to a new model, {\em my-c++}.
	The new code model will show up in the project settings as soon as
	a new View Designer is started. However at the moment the code
	generated with the new {\em my-c++} will be exactly the same as
	created from the {\em c++} model. You will have to change the template
	files to suit your needs.

\section{Template files}
	
	\index{Template file}
	The template files for a code generation model have the name of
	the model as their prefix. All files of the {\em my-c++} model
	therefore have the prefix {\tt my-c++}. The file {\tt my-c++.tmpl}
	is the most important of the template files. Within this file
	the templates for generating an interface are listed. The file
	{\tt my-c++.def.tmpl} is used to create the {\tt InterfaceName.h}
	header file. It should be noted that this template file isn't the
	only one used in creating {\tt InterfaceName.h}. Some parts of
	the file are created using other template files.

	The following list shows which source files are created using which
	template files.
	
	{\tt
	\begin{center}
	\begin{tabular}{r@{\hspace{0.5cm}\arrow\hspace{0.5cm}}l}
	my-c++.rf.tmpl       & InterfaceName.rf \\
	my-c++.def\_base.tmpl & InterfaceName\_\-base.h \\
	my-c++.impl\_base.tmpl& InterfaceName\_\-base.C \\
	my-c++.def.tmpl      & InterfaceName.h \\
	my-c++.impl.tmpl     & InterfaceName.C \\
	my-c++.VcView\_h.tmpl & VcView.h \\
	my-c++.VcView\_C.tmpl & VcView.C \\
	\end{tabular}
	\end{center}
	}

	To create the main program file and the Imakefile other templates
	not mentioned here are used.

\section{Expression Replacement}
	
	\index{Template file!expressions}
	When the View Designer loads a template file, it will search for
	the following expressions. VARNAME is the name of a variable defined
	in the code generation module of the View Designers executable.

	{\tt \$(VARNAME)} -- VARNAME will be replaced by the according string
		in the generated source code.

	{\tt \$(VARNAME,code1,code2)} -- VARNAME is a boolean variable. If
		it's value is true, {\tt code1} will be written to the source.
		If it's value is false, {\tt code2} will be replaced.

	{\tt \$(VARNAME,sub-template)} -- VARNAME is a set of structures.
		For each member of the set the template file
		{\tt sub-template} will be loaded. The expressions in the
		sub template file will be replaced according to the same
		procedures as normal template files.

	Variables may be for example the name of an interface or the resources
	of a widget.

	As shown on page \pageref{tmpl-sample} the generation of source files
	is easy to modify using this method. The file shown is
	{\tt c++.def\_base.tmpl}. This file controls the generation of the
	header file for the base module.
	
	The expression {\tt \$(NAME)} will be replaced by the name of the
	interface. The expression {\tt \$(LINKS,c++.links\_for\-ward.tmpl)}
	results in loading the file {\tt c++.links\_for\-ward.tmpl} and
	expanding all links in the interface.
	
	By creating new code generation models and changing the models
	template files you are allowed to easily extend the View Designers
	abilities and bring your ideas to life. The interface to the
	View Designer are the expressions and replacement rules that were
	just described. All expressions and their replacements will be
	described in the following section.

\section{Interface Description}

	The following notation will be used throughout this section to
	ease the understanding of the code generation interface.
	At the start of each line the name of an expression is given, e.g.
	the description of {\tt \$(CREATION)} will start with {\tt CREATION}.
	To specify the type of the expression the letters {\tt (S)} for a
	string, {\tt (B)} for a boolean and {\tt (A)} for an array will be
	used. If the elements of an array are described, the variable names
	will be separated by a colon from the array name. Example:

	{\tt RESOURCES (A)}\\
	{\tt RESOURCES} is an array of resources for a particular widget.
	For each array element the given {\em sub-template} will be expanded
	and put into the source replacing {\tt RESOURCES}.
	
	{\tt RESOURCES.RESOURCENAME (S)}\\
	The {\tt RESOURCENAME} is a string containing the name of the
	resource, e.g. {\tt labelString}.

	The code generation interface is separated into two parts. The first
	part is the interface to the main program and Imakefile, the second
	is the interface to the view interfaces.

\subsection{Imakefile and main program}

	When generating the source for a project, you may create an Imakefile
	and main program for each interface.

	All uses for expressions mentioned below are related to the already
	existing code generation models C and C++. If you change the code
	generation, you may implement new purposes for them.

	{\tt PROGRAM (S)}\\
	The expression {\tt PROGRAM} is replaced by the name of the programs
	executable file. This name is derived from the name of the project
	file. This expression is mainly used in templates for the Imakefiles
	to create the rules for linking the program.

	{\tt STARTUP (S)}\\
	The startup interface of the project. This interface is opened first
	after starting the application. The startup interface is specified
	in the {\em Settings} entry in the {\em Project} menu. {\tt STARTUP}
	contains the name of the interface, which usually is the name of
	the interface class (C++) or the name of the interface structure (C).
	This expression is used in the main program to create and show the
	startup interface.
	
	{\tt APPL\_CLASS (S)}\\
	This expression contains a string naming the application class of
	the program. This string differs from the program name only in the
	first letter, which will be uppercase. The application class will
	also determine the name for the external resource file. Therefore
	this expression is used in both the Imakefile and the main program.

	{\tt OBJECTS (S)}\\
	A string containing all object files. It is used in the Imakefile
	template. All files have the extension {\tt .o} and are separated
	by spaces.

	{\tt SOURCES (S)}\\
	This expression is used in the Imakefile too. It contains all
	source files of the project. The files have the extension {\tt .c}
	or {\tt .C} and are separated by spaces. The list of the source
	files is normally used in the {\em make depend} rule.

	{\tt RESOURCEFILES (S)}\\
	All public resources are written to files with the extension
	{\tt .rf}. These files are copied to a single resource file
	of the name {\tt \$(APPL\_CLASS).ad}. From this file a header file
	containing the fall-back resources for the project is generated.
	{\tt RESOURCEFILES} contains the name of the files ending with the
	suffix {\tt .rf} separated by spaces.

\subsection{Interfaces}

	For each interface of the project the following expressions are
	known. The generated source files are created using the following
	interface.

	{\tt SHELLCREATION (S)}\\
	This string contains a function call for creating a shell widget. In
	particular this is a call to the {\tt XtCreateWidget} function. The
	parameters to the function call are to be defined before calling this
	function. The {\tt Widget parent} should be set to the parent of
	the created widget. The {\tt ArgList arg} is a list containing all
	resources the widget should be created with. The {\tt Cardinal n}
	is the count of the resources in the {\tt arg} list. Before using
	this expression in a template file, the variables should be set
	and initialized with their appropriate values.
	
	{\tt SHELLIDENTIFIER (S)}\\
	The variable name for the shell widget. Usually this variable is
	set by using the above mentioned function call.

	{\tt SHELLRESOURCES (A)}\\
	This is an array containing all preset resources of a widget. For each
	resource the given file will be loaded and processed. The source
	code segments created in this process will be merged and put into
	the created source template in spite of the expression
	{\tt SHELLRESOURCES}. The default behavior is to set the variable
	{\tt ArgList args} using {\tt XtSetArg()} and to set the variable
	{\tt Cardinal n} to the argument count.

	{\tt SHELLRESOURCES.RESOURCENAME (S)}\\
	This string contains the name of the resource to be set. The name
	may be for example {\tt labelString}. There are two possible ways to
	use this string with {\tt XtSetArg}:

	\begin{enumerate}
	\item by merging it with the prefix {\tt XmN} e.g.
	{\tt XmNlabelString}. This however holds the disadvantage of
	incompatibility with future versions of the View Designer that may
	be using non Motif widget classes, not using the Motif specific
	{\tt XmN} prefix, to enhance functionality.
	\item The string may be enclosed in quotation marks. This has the
	same meaning as the macro {\tt XmNlabelString} but is more portable.
	\end{enumerate}

	{\tt SHELLRESOURCES.RESOURCEVALUE (S)}\\
	The value of the resource to be set is held in this string. This may
	be a C constant, e.g. a number, or a conversion function may be called.
	\begin{enumerate}
	\item {\tt Pixmap createPixmapFromData(char**);}\\
		A pixmap included in the source code will be created.
	\item {\tt Pixmap createPixmapFromFile(char*);}\\
		The pixmap will be loaded at program run time.
	\item {\tt RES\_CONVERT(type, string);}\\
		This is a macro. It converts the resource value given in
		{\tt char *} as {\tt string} to the {\tt type} type. The
		type to convert to will be given as a string, e.g.
		XmString.
	\end{enumerate}


	{\tt NAME (S)}\\
	This expression holds the name of the interface. It can be used as
	class name (e.g. {\tt class \$(NAME)}). It should be noted that the
	name of the interface is the same as the name of its shell widget.

	{\tt LINKS (A)}\\
	All interfaces used as links in an interface are contained in this
	array. Its main purpose is to build header include lists, because
	the the interface of the linked class must be known to the interface
	it is linked to.

	{\tt LINKS.LINK\_CLASS (S)}\\
	The name of the linked interface. With it the preprocessor directive
	{\tt \#include "\$(LINK\_CLASS).h"} can be created.
	
	{\tt WIDGETS (A)}\\
	All widgets of an interface, excluding the shell widget, are held
	in this array. The widgets keep a certain order within the array.
	With the order of the widgets their creation time is determined.
	However not all entries in this array are widgets. The interfaces
	linked to this interface are also among the elements of {\tt WIDGETS}.

	{\tt WIDGETS.ISLINK (B)}\\
	This boolean expression is true if the element is a link.

  	{\tt WIDGETS.MANAGE (B)}\\
	This boolean expression is true if the widget is to be managed. If 
	it's false, the function {\tt XtManageChild} is not to be called.

  	{\tt WIDGETS.IDENTIFIER (S)}\\
	This string holds the name of the widget which will be used as the
	name for its variable too.

	{\tt WIDGETS.TYPE (S)}\\
	The type of the variable. It is normally set to {\tt Widget}, but
	it also may be a pointer to a linked interface.

	{\tt WIDGETS.PARENT (S)}\\
	The name of the parent widget is kept in this string. It is to be
	used to set the variable {\tt Widget parent}.

	{\tt WIDGETS.CREATION (S)}\\
	The creation of the widget or the link is done using the functions
	contained in this string. It is necessary that the variables
	{\tt Widget parent}, {\tt ArgList args} and {\tt Cardinal n} are set
	and verified before using this expression (see also
	{\tt SHELL\-CREATION}). The difference to creation of shell widgets
	are the different functions used to create certain widgets, e.g.
	not only {\tt XtCreateWidget()} but also {\tt XmCreateMenuBar} or
	{\tt createCreditCard()}, if CreditCard as linked interface
	is used.

	{\tt WIDGETS.INTERFACE (S)}\\
	Herein the name of the interface this widget belongs to is contained.
	It can be used to access the class or structure of the interface
	respectively.

	{\tt WIDGETS.RESOURCES (A)}\\
	The resources to be set at creation time are held in this array. It
	is used the same way as the array {\tt SHELLRESOURCES}.

	{\tt WIDGETS.RESOURCES.RESOURCENAME (S)}\\
	The usage of this string is the same as with the according expression
	in the shell widget resources. It should be noted again that the
	resource name is not merged with the prefix {\tt XmN}.

	{\tt WIDGETS.RESOURCES.RESOURCEVALUE (S)}\\
	This expression is also similar to the one used in the shell widget
	resources. Even the conversion routines mentioned there are
	applicable to this string.

	{\tt METHODS (A)}\\
	The callbacks defined for a certain interface are stored in the
	array {\tt METHODS}. It is used to declare the methods and to create
	the callback {\em stubs}.

	{\tt METHODS.METHOD (S)}\\
	The actual name of the method. The string stored in here is the
	same as entered in the widget resource editors callback category.

	{\tt METHODS.INTERFACE (S)}\\
	To avoid naming conflicts or to build the scope of an interface
	class the name of the interface may be used. Within an interface
	the name of a method always must be unique.

	{\tt CALLBACKS (A)}\\
	The array {\tt CALLBACKS} differs from the {\tt METHODS} array in the
	way that in doesn't contain a method element but a element for setting
	a callback. This means that a method may be called for two different
	callback events, thus this array contains at least as many members
	as the {\tt METHODS} array.

	{\tt CALLBACKS.IS\_LINK (B)}\\
	This boolean variable is true if the object the callback is to be set
	for is no widget but a link to another interface.

	{\tt CALLBACKS.IS\_MANAGE\_CB (B)}\\
	Is true if at the time the callback occurs a different widget
	should be managed. This means no method from {\tt METHODS} will be
	called. The function executed will manage the child itself using
	the {\tt XtManageChild()}.

	{\tt CALLBACKS.NAME (S)}\\
	The name of the callback, e.g. {\tt activateCallback}. Note that
	as with the resource names a C String will be formed from this
	name.

	{\tt CALLBACKS.WIDGET\_IDENTIFIER (S)}\\
	The name of the widget the callback is to be set for. This is not
	necessarily a widget. If {\tt IS\_LINK} is true, the string contained
	may be a link. The callback will then be added to the links frame
	widget.

	{\tt CALLBACKS.METHOD (S)}\\
	The name of the method which is to be executed if the callback
	occurs. This does not have to be the same function passed to
	{\tt XtAddCallback()}, for C++ this isn't even possible. The
	function executed would rather be a {\tt static member function}.
	This function will receive a pointer to the instance of the
	interface {\tt this} as {\em client data}. Within the body
	of the function an abstract virtual method will be called which
	is to be implemented by the programmer.

	{\tt CALLBACKS.WIDGET\_TO\_MANAGE (S)}\\
	If {\tt IS\_MANAGE\_CB} is true, this expression holds the name
	of a widget to be managed in case the respective callback occurs.

	{\tt RELATIONS (A)}\\
	A certain kind of resources will be set after all widgets have
	been created. These are in particular resources of the type widget,
	i.e. all resources set with their value being a widget the code
	generation will set after creating all widgets. This is useful
	because all widgets will be present and the resources can be
	set in any order.

	{\tt RELATION.SOURCEWIDGET (S)}\\
	This string contains the name of the widget the resource is to be
	set for.

	{\tt RELATION.RESOURCENAME (S)}\\
	The name of the widget resource to be set. It can be e.g. of the
	type {\tt leftWidget} or {\tt subMenuId}. The prefix {\tt XmN} must
	be taken care of here too.

	{\tt RELATION.DESTINATIONWIDGET (S)}\\
	The widget being set as resource, e.g. the widget the resource widget
	should align to if both are children of a XmForms widget and the
	resource {\tt leftWidget} is set.

	{\tt RELATION.SOURCE\_ISLINK (B)}\\
	{\tt RELATION.DESTINATION\_ISLINK (B)}\\
	The objects referenced can be links of course, in which case the
	respective frame widget is to be used.

	{\tt HEADERS (A)}\\
	All widgets used in a certain interface need different header files.
	This array contains all header files necessary for the interface.

	{\tt HEADERS.HEADER (S)}\\
	The relative path name of the header file to be used in the
	preprocessor {\tt \#include} statement (e.g.
	{\verb!#include <$(HEADER)>!}).

	{\tt LOCALHEADERS (A)}\\
	With the inclusion of Xpm files additional header files are
	necessary. All necessary header files are present in this array.
	These headers are identical to the Xpm files to be included.

	{\tt LOCALHEADERS.HEADER (S)}\\
	The name of the Xpm file.

	{\tt RESOURCEFILE (S)}\\
	If resources are set to {\em public} in the widget resource editor
	then it is to be set in an external resource file. All public
	resources of an interface will be written to a file with the
	extension {\tt .rf} using this expression.

\begin{figure}
\begin{maxipage}
\begin{small}
	{\baselineskip0.4cm \label{tmpl-sample}
	\begin{verbatim}
	// -----------------------------------------------------
	// $(NAME)_base.h
	// 
	// This file was automaticly generated by
	// View Designer/X                          
	//
	// Don't edit this file, please !
	// -----------------------------------------------------

	#ifndef _$(NAME)_base_H_
	#define _$(NAME)_base_H_
	
	#include <X11/Intrinsic.h>
	#include "VcView.h"
	
	$(LINKS,c++.links_forward.tmpl)
	
	// -----------------------------------------------------
	// Class definition 
	// class $(NAME)_base is a derived class of VcView
	// -----------------------------------------------------
	
	class $(NAME)_base : public VcView {
	private:
	  void createWidgets();
	  void addRelations();
	  void addCallbacks();
	
	protected:
	  Widget $(SHELLIDENTIFIER);
	$(WIDGETS,c++.widgetdeclaration.tmpl)
	
	$(METHODS,c++.static_cb_declaration.tmpl)
	
	
		  // override the following methods in class $(NAME)
	
	$(METHODS,c++.cb_abstract_declaration.tmpl)
	
	public:
	  $(NAME)_base(Widget parent = NULL, ArgList args = NULL, 
	                                     Cardinal n = 0);
	
	  Widget getTop() { return $(SHELLIDENTIFIER); }
	  Pixmap createPixmapFromData(char **data);
	  Pixmap createPixmapFromFile(char *filename);
	  static void manageWidget(Widget wg, XtPointer c,
	                                      XtPointer call);
	};
	
	#endif /* Don't add stuff after this #endif */
	\end{verbatim}
	}
\end{small}
\end{maxipage}
\caption{Example for a template file}
\end{figure}

\lazyglossar

\myitem{Dialog}
	Part of a user interface which is only visible at certain times
	during application execution. Dialogs are used to show special
	information to the user or to present the user with choices.

\myitem{Drag \& Drop}
	Method used in graphical user interfaces to facilitate copying,
	moving or tieing objects using the pointing device (usually the
	mouse). The OSF/Motif user interface implements this method
	using the middle mouse button.

\myitem{Imake}
	A package to create Makefiles for X applications. The package
	is contained with the standard X Window System distribution. The
	source the Imake package uses to create the makefiles is a very
	simple Imakefile.

\myitem{Option Menu}
	Part of a user interface used to select one of a set of possible
	options. Only the currently selected choice is visible. By clicking
	with the mouse on the menu it will expand to show all possible
	options to choose from.

\myitem{OSF} 
	Short for Open Software Foundation. A consortium of firms with the
	aim to establish standards for open systems. Products of the OSF
	are e.g. Motif, DCE and OSF/1.
	
\myitem{OSF/Motif}
	Toolkit based on the X Window System. It defines many widget classes
	and functions to create graphical user interfaces.

\myitem{Push Button}
	Part of a user interface that performs an action if clicked with
	the mouse or activated via the keyboard. 

\myitem{Shell Widget}
	Implements the interface to the world outside the widget
	(Window Manager). The shell widget and its children appear in
	a certain window controlled by the window manager.

\myitem{Toggle Button}
	A widget that, if activated by mouse or keyboard, switches between
	two states, e.g. on and off.

\myitem{User Interface}
	The interface through which an application interacts with the user.

\myitem{Widget}
	Object of an user interface. Every widget has its distinctive look
	and behavior. All of them are controlled by resources. Widgets
	are for example text entry fields or menus.

\myitem{Widget Class}
	A widget class defines the characteristics of a widget. The widget
	classes form a hierarchy which base is the class Core.

\myitem{Window Manager}
	A program that manages and controls the windows of applications
	running on a machine (e.g. changing size, window movement and so on).
	The window manager published by the OSF is called {\bf mwm}.

\myitem{X Window System}
	Network transparent window system for open systems. The X Window
	System is the base for designing and implementing graphical user
	interfaces for most Unixes. It is developed by the Massachusetts
	Institute of Technology (MIT).

\lazyindex

\end{document}
% end of document
% don't add stuff after this 
  