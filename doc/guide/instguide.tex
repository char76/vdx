\documentstyle[epsf,palatino,umlaut,twoside,german,refman]{article} 

\sloppy
\parindent0.0cm		% Keine Einrueckung beim Absatzbeginn

\pagestyle{headings}

\title{
	View Designer/X 1.0 - Installationshinweise
}

\author{
	Dirk L�ssig
}

% ------------------------------------
% Beginn des Dokumentes
% ------------------------------------

\begin{document}

\maketitle

\thispagestyle{empty}


\section{Installation der Dateien}

	Auf der mitgelieferten Diskette befindet sich der View Designer/X
	als komprimiertes Tar-Archiv. Sie k�nnen den VDX an einer beliebigen
	Stelle im Dateisystem installieren. 

	\begin{itemize}

	\item Legen Sie die Installationsdiskette in das Diskettenlaufwerk
	ein.

	\item Wechseln Sie in ein Verzeichnis, in das Sie den VDX installieren
	wollen. Ein solches Verzeichnis k�nnte z.B. das Verzeichnis
	{\tt /usr/local} oder {\tt /opt} sein.

	\item Geben Sie folgendes Kommando ein, um die Dateien des VDX von
	der Diskette in das Dateisystem zu �bertragen:\par
	{\tt tar xzvf /dev/fd0}\par
	Falls die Diskette nicht im ersten Laufwerk ({\tt /dev/fd0}) liegt,
	modifizieren Sie das Kommando.

	\item Das Tar-Kommando erzeugt ein neues Verzeichnis, in das Sie
	nun wechseln und das Shell-Skript {\tt ./install.sh} ausf�hren.
	
	\end{itemize}

	Im {\tt bin}-Verzeichnis befindet sich das ausf�hrbare Programm
	{\tt vdx}, mit dem Sie den View Designer/X starten k�nnen.
	Der VDX l�uft dabei noch im Demonstationsmodus, da keine
	Lizenzinformationen installiert wurden.

\section{Installation der Lizenznummer}

	Haben Sie von Ihrem H�ndler das Produkt VDX gekauft, so hat er
	Ihnen eine Lizenznummer mitgeteilt. Diese Lizenznummer
	ist mit dem Programm {\tt addVdxLicense} zum VDX hinzuzuf�gen. Erst
	danach kann der View Designer im normalen Modus betrieben werden.
	Das Programm {\tt addVdxLicense} befindet sich im {\tt
	bin}-Verzeichnis des installierten Dateibaumes. Geben Sie folgendes
	Kommando ein:\par
	{\tt addVdxLicense Lizenznummer}\par
	Dabei stellt {\tt Lizenznummer} die Zahlenkombination dar, die
	Ihnen vom H�ndler �bermittelt wurde.


\section{Registrierkarte}

	Wir hoffen, da� Ihnen der View Designer/X ein gro�e Hilfe bei
	Ihrer Entwicklungsarbeit ist. Damit wir Ihnen auch weiterhin ein
	gutes Produkt anbieten k�nnen, bitten wir Sie, Vorschl�ge und
	Kommentare an uns zu senden.

	{\bf Bitte vergessen Sie nicht die Registrierkarte auszuf�llen und
	abzuschicken.}

\end{document}




